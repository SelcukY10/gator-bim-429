\section{Giriş}

\indent Dizi sektörü, Türkiye'nin en önemli kültürel ve ekonomik sektörlerinden biridir. Her yıl onlarca yeni dizi yayına girmekte ve bu diziler milyonlarca izleyici tarafından takip edilmektedir. Ancak, yeni bir dizinin başarılı olup olamayacağı her zaman kesin olarak bilinememektedir. Bazen çok iyi kadroya sahip bir dizi yayına girdikten sonra ilgi görmeyebilirken, bazen hiç rağbet görmeyeceği düşünülen bir dizi büyük bir başarı yakalayabilir. \\
\indent Bu projede, veri madenciliği teknikleri kullanılarak yeni bir Türk dizisinin devamlılığının olup olamayacağına dair tahmin yapılması amaçlanmaktadır. Bu amaçla, geçmiş yıllarda yayınlanan Türk dizilerinin verileri incelenecek ve bu verilerden elde edilen sonuçlar kullanılarak bir tahmin modeli geliştirilecektir. \\
\indent \textbf{Projenin ilk aşamasında}, yeni bir dizinin devamlılığını etkileyebilecek faktörler belirlenecektir. Bu faktörler, daha önce yapılan araştırmalara dayanılarak belirlenecektir. \\
\indent \textbf{İkinci aşamada}, belirlenen faktörlerin geçmiş yıllarda yayınlanan Türk dizilerine etkisi incelenecektir. Bu amaçla, dizilerin yayın tarihleri, oyuncu kadrosu, senaryo, TV kanalı ve yayınlanma günü gibi veriler toplanacak ve bu veriler kullanılarak faktörlerin dizilerin devamlılığına etkisi analiz edilecektir.\\ 
\indent \textbf{Üçüncü aşamada}, ikinci aşamada elde edilen sonuçlar kullanılarak bir tahmin modeli geliştirilecektir. Bu model, yeni bir dizinin devamlılığını etkileyebilecek faktörleri dikkate alarak, dizinin devamlılığının olup olamayacağına dair bir tahminde bulunacaktır. \\
Bu projenin sonucunda, yeni bir Türk dizisinin devamlılığını tahmin etmek için kullanılabilecek bir model geliştirilmesi hedeflenmektedir. Bu model, dizi sektöründe önemli bir ihtiyaç olan dizilerin başarısının önceden tahmin edilmesini mümkün kılacaktır.